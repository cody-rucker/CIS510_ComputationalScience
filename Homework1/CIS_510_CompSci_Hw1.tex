\documentclass[12pt]{article}

\usepackage{graphicx}
\usepackage{amsfonts}
\usepackage{amssymb}
\usepackage{amsmath}
\usepackage{mathtools}
\usepackage{color}
\usepackage{changepage}
\usepackage[margin=1.2in]{geometry}
\newcommand{\Tau}{\mathrm{T}}
\usepackage{enumitem}
\usepackage{subfig}
\usepackage{enumitem}
\usepackage{sectsty}
\usepackage{upgreek}
\usepackage{setspace}
\usepackage{cite}
\usepackage[final]{pdfpages}
\usepackage{float}
\usepackage{siunitx}
\usepackage{wrapfig}
\usepackage{tikz}
\usetikzlibrary{decorations.pathreplacing}
\usetikzlibrary{arrows,angles,quotes}
\usepackage{systeme}
\usepackage{listings}
\usepackage{scrextend}
\usepackage{minted}


\sectionfont{\fontsize{15}{18}\selectfont}
\subsectionfont{\fontsize{12}{15}\selectfont}

\begin{document}

\begin{minted}{julia}

using LinearAlgebra

"""
    ComputeLU(a)

Compute and return LU factorization of sqaure matrix a.

# Examples
'''
julia> A = rand(3, 3)
julia> (L, U) = ComputeLU(A)
'''
"""


function ComputeLU(A)
      N = size(A)[1]

      Id  = Matrix{Float64}(I, N, N)
      ℓ   = copy(Id)
      ℓ⁻¹ = copy(Id)
      Ã   = copy(A)
      L   = copy(Id)


      for k = 1:N-1
            ℓ   .= Id
            ℓ⁻¹ .= Id


            for i = k+1:N
                  ℓ[i, k]   = -Ã[i,k] / Ã[k,k]
                  ℓ⁻¹[i, k] = Ã[i,k] / Ã[k,k]
            end

            Ã .= ℓ * Ã
            L .= L * ℓ⁻¹
      end

      U = Ã

      return L, U
end


N = 100
A = Array{Float64}(undef, N, N)
A .= rand(N, N)

(myL, myU) = ComputeLU(A)

@assert myL*myU ≈ A


\end{minted}

\end{document}