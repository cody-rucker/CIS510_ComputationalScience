\documentclass[12pt]{article}

% needed to use minted with julia (compile with LuaLaTeX
\usepackage{lineno}
\usepackage{fontspec}
\usepackage{polyglossia}
\setmonofont{DejaVu Sans Mono}[Scale=MatchLowercase]
\usepackage{minted}
\usepackage{latexsym, exscale, stmaryrd, amsmath, amssymb}
\usepackage{unicode-math}
%----------------------------------------------------------

\usepackage{graphicx}
\usepackage{amsfonts}
\usepackage{amssymb}
\usepackage{amsmath}
\usepackage{mathtools}
\usepackage{color}
\usepackage{changepage}
\usepackage[margin=1in]{geometry}
\newcommand{\Tau}{\mathrm{T}}
\usepackage{enumitem}
\usepackage{subfig}
\usepackage{enumitem}
\usepackage{sectsty}
\usepackage{upgreek}
\usepackage{setspace}
\usepackage{cite}
\usepackage[final]{pdfpages}
\usepackage{float}
\usepackage{siunitx}
\usepackage{wrapfig}
\usepackage{tikz}
\usetikzlibrary{decorations.pathreplacing}
\usetikzlibrary{arrows,angles,quotes}
\usepackage{systeme}
\usepackage{listings}
\usepackage{scrextend}

\usepackage{tabu}


\makeatletter
\renewcommand{\fnum@figure}{Fig. \thefigure}
\makeatother

\sectionfont{\fontsize{15}{18}\selectfont}
\subsectionfont{\fontsize{12}{15}\selectfont}

\usepackage{fancyhdr}

\pagestyle{fancy}
\fancyhf{}
\rhead{Cody Rucker}
\lhead{CIS 510: Assignment 2}
\rfoot{\thepage}

\begin{document}
\begin{flushleft}
\par \textbf{1.} We solve the heat equation 
$$u_t = ku_{xx}$$
on the spatial domain $(0,1)$ and time domain $(0,T)$. We specify and initial position $u(x,0) = f(x)$ and impose homogeneous Dirichlet conditions at the boundaries $x=0, x=1$. Convergence tests are run on the manufactured solution $u(x,t) = e^{-2t}\sin{\pi x}$ with $k=2$, $T=0.1$, and $\lambda = 0.1$. Figure 1 is a table containing the resulting convergence data as the mesh is sequentially refined. As expected, the numerical solution exhibits rate 2 convergence. 
\begin{figure}[h]
\captionsetup{width=.5\linewidth}
\center
{\tabulinesep=1.5mm 
\begin{tabu}{|| c | c | c ||} 
\hline \hline 
      $h$ & $ \left \lVert u-u_h \right \rVert $ & $ \log_2 \big (e_h /e_{\frac{h}{2}} \big ) $ \\  \hline 
  0.10000 &            0.000092 &       $ \emptyset $ \\ \hline  
  0.05000 &            0.000023 &             2.00765 \\ \hline  
  0.02500 &            0.000006 &             2.00191 \\ \hline  
  0.01250 &            0.000001 &             2.00048 \\ \hline  
\hline 
\end{tabu}}
\caption{Convergence of the numerical solution for $T=0.1, k=2, \lambda =0.1$. }
\end{figure}


\section*{2.} Set $T=2$ and $\Delta x = 0.1$. Stability of forward Euler is dictated by the condition 
$$k\frac{\Delta t}{\Delta x^2} \leq 1.$$
Then we must have that $200\Delta t \leq 1$. For the time steps $0.1, 0.01, 0.001$ we can see that only one of these steps ($\Delta t = 0.001$) actually satisfies the stability requirement. When plotting the solution using the other two time steps we see the numerical solution blow up.

\section*{3.} Set $T=2$ and $\Delta x = 0.1$. Unlike forward Euler, the backward Euler method is unconditionally stable. So regardless of the choice for $\Delta t$, the backward Euler method gives an approximation to the exact solution at each time. Although I did not explore this I would be curious to look at the scenario in which backward Euler's computation cost hinder it and how fast it runs in comparison to forward Euler. 
\end{flushleft}

\pagebreak
\section*{heat1D.jl}

\begin{minted}[xleftmargin=20pt, linenos]{julia}
using SparseArrays
using Plots
include("myForwBack_Euler.jl")
include("/home/cody/github/Finite_Element_Methods/convergence_rates.jl")

# source function
function F(x,t)
    return 2*(pi^2 -1)*exp(-2t)*sin.(pi*x)
end

# initial value function
function f(x)
    return sin.(pi*x)
end

# exact solution
function exact(x,t)
    return exp(-2t)*sin.(pi*x)
    #return exp(-pi^2 * t)*sin.(pi*x)
end

function time_dependent_heat(k, Δx, Δt, T, my_source, my_initial, my_exact)
    N  = Integer(ceil((1-0)/Δx)) # N+1 total nodes, N-1 interior nodes
    x = 0:Δx:1
    t = 0:Δt:T
    M = Integer(ceil((T-0)/Δt)) # M+1 total temporal nodes

    λ = Δt/Δx^2

    # A is N-1 by N-1 matrix for forward Euler
    A = (1-2*λ*k)*sparse(1:N-1,1:N-1,ones(N-1), N-1, N-1) +
            (λ*k)*sparse(2:N-1,1:N-2,ones(N-2),N-1,N-1) +
            (λ*k)*sparse(1:N-2,2:N-1,ones(N-2),N-1,N-1)
	
	# A2 is the appropriate matrix for backward Euler
    A2 = (1+2*λ*k)*sparse(1:N-1,1:N-1,ones(N-1), N-1, N-1) -
             (λ*k)*sparse(2:N-1,1:N-2,ones(N-2),N-1,N-1) -
             (λ*k)*sparse(1:N-2,2:N-1,ones(N-2),N-1,N-1)

    u = Array{Float64}(undef,N-1)
    u .= my_initial(x[2:N])  # setting the initial condition for interior nodes
    u2 = copy(u)
    Y = zeros(N+1,M)         # initializing solution matrix
    Z = copy(Y)

    my_forward_Euler!(Y, Δt, t, A, F, u, x, N, M)
    my_backward_Euler!(Z, Δt, t, A2, F, u2, x, N, M)


    uₑ₁ = my_exact(x[:], t[M]) - Y[:, M]
    uₑ₂ = my_exact(x[:], t[M-1]) - Z[:, M-1]

    e1 = sqrt(Δx * uₑ₁' * uₑ₁)
    e2 = sqrt(Δx * uₑ₂' * uₑ₂)

    return Y, Z, e1, e2
end

# repackage the solution as a function of grid spacing which returns the discrete
# L²-error
function R(Δx)
    k = 2
    #Δx = 0.1

    T = 2
    λ = 0.2
    Δt = λ*Δx^2 / k

    x = 0:Δx:1
    t = 0:Δt:T

    U1, U2, e1, e2 = time_dependent_heat(k, Δx, Δt, T, F, f, exact)
return U1, U2, e1, e2
end
\end{minted}

\pagebreak

\section*{your\textunderscore ForwBack\textunderscore Euler}
(I just modded your code)
\begin{minted}[xleftmargin=20pt, linenos]{julia}
using LinearAlgebra
using Plots

function my_forward_Euler!(Y, Δt, t, A, F, y, x, N, M)
    #M = length(t)
    #t = t1:Δt:tf
    #M = Integer(ceil((tf-t1)/Δt))
    Y[2:N,1] = y[:]

    for n = 1:M
        b = Δt * F(x[2:N],t[n])
        y[:] = A * y[:] + b
        Y[2:N,n] .= y[:]

    end

    #return (t, Y)

end


function my_backward_Euler!(Y, Δt, t, A, F, y, x, N, M)


    Y[2:N,1] = y[:]

    #Exact = Matrix{Float64}(undef,2,N+1)
    #Exact[:,1] = y[:]

    Id = I(N-1) #Matrix{Float64}(I,N+1,N+1)

    for n = 2:M-1
        y[:] = A\(y[:] .+ Δt*F(x[2:N], t[n]))
        Y[2:N,n] = y[:]
        #Exact[:,n] = exact(t[n])
    end
end
\end{minted}
\end{document}